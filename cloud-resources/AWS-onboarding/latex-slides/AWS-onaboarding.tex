% ---------------------------- Preamble starts here ----------------------------

\documentclass[aspectratio=169]{beamer} %Remove [aspectratio=169] to get non-wide 4:3 slide aspect ratio

%-----------------------------------------------
% --- Set beamer theme
\usetheme{Metropolis}
\setbeamertemplate{footline}{}				% Remove automatic footer
\setbeamertemplate{navigation symbols}{}	% Comment this line to display navigation symbols

%-----------------------------------------------
% Load i2i symbol
\addtobeamertemplate{frametitle}{}{%
\begin{textblock*}{\linewidth}(0cm,7.4cm) % Replace with (0cm, 8cm) if using non-wide slide aspect
	\includegraphics[width=\linewidth]{img/Footer.png}
\end{textblock*}}

\setbeamertemplate{footline}{\hfill\insertframenumber/\inserttotalframenumber}

%-----------------------------------------------
% --- Load packages
\usepackage{textpos}		% To align objects correctly
\usepackage{multicol}		% To right in multiple columns
\usepackage{color}			% To color text

%-----------------------------------------------
% --- Include link to last commit
\usepackage{xstring}
\usepackage{catchfile}

%Set this user input
\newcommand{\gitfolder}{../../../.git} %relative path to .git folder from .tex doc
\newcommand{\reponame}{worldbank/dime-github-trainings} % Name of account and repo be set in URL

%Based on this https://tex.stackexchange.com/questions/455396/how-to-include-the-current-git-commit-id-and-branch-in-my-document
\CatchFileDef{\headfull}{\gitfolder/HEAD.}{} 				%Get path to head file for checked out branch
\StrGobbleRight{\headfull}{1}[\head]						%Remove end of line character
\StrBehind[2]{\head}{/}[\branch]							%Parse out the path only
\CatchFileDef{\commit}{\gitfolder/refs/heads/\branch.}{}	%Get the content of the branch head
\StrGobbleRight{\commit}{1}[\commithash]					%Remove end of line characted

%Build the URL to this commit based on the information we now have
\newcommand{\commiturl}{\url{https://github.com/\reponame/commit/\commithash}}

%-----------------------------------------------
% --- Add your information here
\title{GitHub - Pull Request training}
\author{DIME Analytics}
\institute{DIME - The World Bank - \trainingURL{https://www.worldbank.org/en/research/dime}}
\date{\today}

\newcommand{\trainingURL}[1]{{\color{blue}\url{#1}}}

\newcommand{\traininerUsername}{kbjarkefur}
\newcommand{\repoName}{\traininerUsername/lyrics}
\newcommand{\trainingRepoURL}[1]{\trainingURL{github.com/\repoName #1}}
\newcommand{\trainerEmail}{\trainingURL{kbjarkefur@worldbank.org} }


% ---------------------------- Preamble ends here ----------------------------

\begin{document}

\begin{frame}
\includegraphics[width=\textwidth]{img/Header.png}
\vspace{-0.2cm}
\titlepage 	 % Opening slide, prints inform
\end{frame}


\section{WB AWS}





\begin{frame}
\frametitle{A typical WB AWS resource}

	\begin{columns}[c]
		\column{.50\textwidth} % Left column and width
		\large All WB AWS resources will be hosted in an \textit{AWS Island}
		\vspace{.7cm}\newline
		\large \textbf{Manage}: All WB AWS resources are managed through the AWS console.
		\vspace{.7cm}\newline
		\large \textbf{Access}: Any access to a WB AWS resource must go trough a gateway (gw) server
		
		\column{.50\textwidth} % Right column and width
		\begin{figure}
			\centering
			\includegraphics[width=\textwidth]{./img/wb-aws.png}
		\end{figure}

	\end{columns}
\end{frame}



\section{Manage WB AWS resources}

\begin{frame}
	\frametitle{WB AWS resource - Manage?}
	\begin{columns}[c]
		\column{.50\textwidth} % Left column and width
		We \textbf{manage} the settings of our AWS resources in the AWS console. 
		\vspace{.5cm}\newline
		Mostly relevant for ITS, but a few settings are important to us
		\vspace{.5cm}\newline
		Examples:
		\begin{itemize}
			%\setlength\itemsep{1em}
			\item Start and stop EC2 instances (cost efficiency)
			\item Manually upload files to S3 buckets
		\end{itemize}
		
		\column{.50\textwidth} % Right column and width
		\begin{figure}
			\centering
			\includegraphics[width=\textwidth]{./img/wb-aws.png}
		\end{figure}
		
	\end{columns}
\end{frame}

\begin{frame}
	\frametitle{Manage resource - set-up 1/2}
	\begin{columns}[c]
		
		\column{.60\textwidth} % Right column and width
		
		To manage WB AWS resources we need a \textit{C-account} - C as in cloud
		
		\begin{enumerate}
			\item Request C-account on eServices
			\item Reset auto-generated C-account password:
			\begin{enumerate}
				\item Go to \textit{http://password/} on WB intranet
				\item Log in using SecureID and select \textit{Manage Passwords}
				\item Click \textit{Change} for your \textit{Cloud Admin} account 
			\end{enumerate}

		\end{enumerate}
		
		\column{.40\textwidth} % Right column and width
		\begin{figure}
			\centering
			\includegraphics[width=.5\textwidth]{./img/password-1.png}
		\end{figure}
		\vspace{.2cm}
		\begin{figure}
			\centering
			\includegraphics[width=1\textwidth]{./img/password-2.png}
		\end{figure}
		
	\end{columns}
\end{frame}

\begin{frame}
	\frametitle{Manage resource - set-up 2/2}
	\begin{columns}[c]
		
		\column{.650\textwidth} % Right column and width
		
		\begin{enumerate}
			\item Download the \textit{Microsoft Authenticator} app to your smartphone
			\item Go to \url{cloudportal/} on WB intranet
			\item Log in using \textit{c$<$UPI$>$@worldbankgroup.org} (your C-account) and your new password
			\item Click next until you see a QR code - open \textit{Microsoft Authenticator} and scan the QR code
			\item Use code, fingerprint, pattern or similar to approve the authentication in the app
			\item Follow the instructions for how to test the setup, and if success you should see "Verification Successful"
			
		\end{enumerate}
		
		\column{.35\textwidth} % Right column and width
		\begin{figure}
			\centering
			\includegraphics[width=1\textwidth]{./img/microsoft-auth-1.png}
		\end{figure}
		
	\end{columns}
\end{frame}

\begin{frame}
	\frametitle{Manage resource - log on}
	
	
	\begin{columns}[c]
		
		\column{.55\textwidth} % Right column and width
		
		When you are set up, this is how you access the AWS console each time:
		
		\begin{enumerate}
			\item Go to \url{cloudportal/} on WB intranet
			\item Log in using \textit{c$<$UPI$>$@worldbankgroup.org} (your C-account) and your new password
			\item Authenticate with \textit{Microsoft Authenticator}
			\item Click "Amazon web services"
			\item Clikc "AWS Account", then expand (red square) and then click "Management Console"
			
		\end{enumerate}
		
		\column{.45\textwidth} % Right column and width
		\begin{figure}
			\centering
			\includegraphics[width=.5\textwidth]{./img/logon-1.png}
		\end{figure}
		\vspace{.2cm}
		\begin{figure}
			\centering
			\includegraphics[width=1\textwidth]{./img/logon-2.png}
		\end{figure}
	\end{columns}
\end{frame}


\begin{frame}
	\frametitle{Manage resource - start an EC2 instance}
	
	\begin{columns}[c]
		
		\column{.55\textwidth} % Right column and width
		
		Start an EC2 instance so that you can remote access into it:
			
		\begin{enumerate}
			\item Search for \textit{EC2}
			\item In the menu to the left click \textit{Instances} - see below
			\item Check the instance you want to start - click \textit{Instance state} and select \textit{Start instance}
			\item Refresh until \textit{Pending} turns into \textit{Running}
			
		\end{enumerate}
		
		\begin{figure}
			\centering
			\includegraphics[width=.4\textwidth]{./img/ec2-1.png}
		\end{figure}
		
		
		\column{.45\textwidth} % Right column and width
		\begin{figure}
			\centering
			\includegraphics[width=1\textwidth]{./img/ec2-2.png}
		\end{figure}
		\begin{figure}
			\centering
			\includegraphics[width=1\textwidth]{./img/ec2-3.png}
		\end{figure}
		\vspace{.5cm}
		
	\end{columns}
\end{frame}


\section{Access WB AWS resources}

\begin{frame}
	\frametitle{WB AWS resource - Access?}
	\begin{columns}[c]
		\column{.50\textwidth} % Left column and width
		\textbf{Access} through the gateway server is both when a resource access the internet and when a resource is accessed from the internet. 
		\vspace{.5cm}\newline
		Examples:
		\begin{itemize}
			%\setlength\itemsep{1em}
			\item Remote in to run a script on an EC2 server
			\item Browsing a shiny app hosted on WB AWS
			\item A script hosted in WB AWS scraped the internet
		\end{itemize}
		
		\column{.50\textwidth} % Right column and width
		\begin{figure}
			\centering
			\includegraphics[width=\textwidth]{./img/wb-aws.png}
		\end{figure}
		
	\end{columns}
\end{frame}


\section{WB AWS best practices}

\end{document}
