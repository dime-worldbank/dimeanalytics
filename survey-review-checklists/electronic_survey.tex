\documentclass{report}
\usepackage[left=2cm, right=2cm, top=2cm]{geometry}
\usepackage[most]{tcolorbox}

\usepackage{nopageno}
\usepackage{fancyhdr}
\usepackage{adjustbox}

\usepackage{booktabs,tabularx}
\usepackage{hyperref}
%\usepackage{showframe}
\usepackage{xcolor}
\usepackage{graphics}
\pagestyle{fancy}
\fancyhf{}
\renewcommand{\headrulewidth}{0pt}
\lfoot{\includegraphics[height=1cm,keepaspectratio]{img/i2i}}
\cfoot{\includegraphics[height=1cm,keepaspectratio]{img/wb}}
\rfoot{\includegraphics[height=1cm,keepaspectratio]{img/analytics}}
\definecolor{fontcolor}{HTML}{7A0569}
\newcommand{\titleBox}[1]{
	\begin{tcolorbox}
		[colframe = fontcolor,
		colback = fontcolor,
		sharp corners,
		halign = flush center,
		valign = center,
		height = 0.2\textwidth,
		after skip = 1cm]
		#1
	\end{tcolorbox}
}


% ------------------------------ End of preamble ---------------------------------------------
\begin{document}


\titleBox{
	\textcolor{white}{\Large{Project: Name}} \\
	\textcolor{white}{\Large{Survey: Round}} \\
	\textcolor{white}{\Large{DIME Analytics Electronic Survey Review Checklist}}
}

This report details the findings from review of the electronic survey for the \textit{Survey Round}  of the \textit{Project name} project.
\section*{Main findings}

\subsection*{Required practices}
\begin{itemize}
	\item \textit{List required practices identified for project}
\end{itemize} 
\subsection*{Recommended practices}

The \href{https://dimewiki.worldbank.org/Ietestform}{ietestform} command (from the iefieldkit package) was run and the results are attached. It is a good practice to run iestestform multiple times during survey development. This is to make sure you’ve incorporated all suggested changes that are relevant and to re-test when new questions are added during the development phase of the survey.
\begin{itemize}
	\item \textit{List recommended practices identified for project}
\end{itemize}

The checklist in the next section details the exact findings.
\newpage 

\section*{Checklist}

		\subsection*{Encryption}

\newcounter{row}

\newcommand{\makerow}[1]{%
 % #1 = text field
 #1 &
 \stepcounter{row}%
 \mbox{\CheckBox[print,name=YES\therow, width=0.7em, height=0.7em]{}} &
 \mbox{\CheckBox[print,name=NO\therow, width=0.7em, height=0.7em]{}} &
 \mbox{\CheckBox[print,name=N/A\therow, width=0.7em, height=0.7em]{}} \\
}


\begin{Form}
\noindent
\begin{tabularx}{\textwidth}{Xccc}
\toprule
\textbf{CHECKS} & YES & NO & N/A \\
\midrule
\makerow{Survey form \textbf{encrypted}}
\midrule
\makerow{Survey form distinguish \textbf{publishable and non-publishable} fields}
\bottomrule
\end{tabularx}

		\subsection*{Ensuring data quality}

\noindent
\begin{tabularx}{\textwidth}{Xcccc}
\toprule
\textbf{CHECKS} & YES & NO & N/A \\
\midrule
\makerow{The survey incorporates \textbf{validation of key variables} (ID number, phone number etc)}
\midrule
\makerow{Answer choices \textbf{systematically coded} (ie, -9 for "Other", -8 for "Do not know", and -7 for "Prefer not to answer”)}
\midrule
\makerow{The survey makes use of \textbf{constraints} on numeric variables to minimize entry errors}
\midrule
\makerow{The survey has appropriate \textbf{skip patterns and relevance} (ie, the question “What level of schooling are you in?” is only relevant if the question “Are you in school” is answered Yes)}
\midrule
\makerow{Potentially required \textbf{calculations} carried out within the survey instrument}
\bottomrule
\end{tabularx}


\vspace{5mm} %5mm vertical space
\subsection*{Form usability}

\noindent
\begin{tabularx}{\textwidth}{Xcccc}
	\toprule
	\textbf{CHECKS} & YES & NO & N/A \\
	\midrule
	\makerow{Survey form is \textbf{translated} to local language}
	\midrule
	\makerow{If translated, \textbf{supplemental csv’s} is in UTF-8 format (to ensure special characters in translation are shown correctly)}
	\midrule
	\makerow{\textbf{Combined different modules} into clearly defined groups for easy viewing}
	\midrule
	\makerow{\textbf{Includes notes} throughout to clearly guide enumerators/respondents}
	\bottomrule
\end{tabularx}

\newpage 
\subsection*{ietestform results}
The following checks are identified by the report generated by running the \textbf{ietestform} command.  
\vspace{5mm} %5mm vertical space

\noindent
\begin{tabularx}{\textwidth}{Xcccc}
\toprule
\textbf{CHECKS} & YES & NO & N/A \\
\midrule
\makerow{Non-note type fields that are  required}
\midrule
\makerow{Note type fields that are not required unless required}
\midrule
\makerow{“begin group” and “end group” fields match}
\midrule
\makerow{“begin repeat” and “end repeat” fields match}
\midrule
\makerow{Field names valid for Stata import}
\midrule
\makerow{Field labels valid for Stata import}
\midrule
\makerow{Choice labels valid for Stata import}
\midrule
\makerow{Field labels of appropriate length for Stata import}
\midrule
\makerow{Field names of appropriate length for Stata Import}
\midrule
\makerow{Field names are unique}
\midrule
\makerow{Fields without Leading or Trailing spaces}
\midrule
\makerow{No duplicate Labels in Choice list}
\midrule
\makerow{Only numeric value in choice name or value}
\midrule
\makerow{No unused choice lists}
\midrule
\makerow{No duplicate Values in Choice list}
\midrule
\makerow{No missing label, value or name of choice list}
\midrule
\makerow{Latest version of SurveyCTO syntax is used}
\bottomrule
\end{tabularx}



\end{Form}


\end{document}

